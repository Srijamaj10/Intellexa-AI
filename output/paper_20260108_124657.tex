
\\documentclass{article}
\\usepackage[utf8]{inputenc}
\\usepackage{amsmath}
\\usepackage{amssymb}
\\usepackage{graphicx}
\\usepackage[T1]{fontenc}
\\usepackage{hyperref}

\\title{Adaptive Visible Light Communication Data Rate and Modulation Schemes for Batteryless Internet of Things Systems}
\\author{AI Research Assistant}
\\date{January 26, 2026}

\\begin{document}

\\maketitle

\\begin{abstract}
Batteryless Internet of Things (IoT) devices powered by ambient energy sources, such as solar panels, offer significant advantages in terms of longevity and reduced maintenance. Visible Light Communication (VLC), leveraging the same solar panel for both energy harvesting and data reception, presents an energy-efficient communication paradigm for these devices. However, the performance of fixed-rate VLC systems is highly susceptible to variations in ambient light conditions, distance, and energy availability, leading to suboptimal throughput or communication failures. This paper proposes a novel framework for adaptive VLC data rate and modulation schemes specifically tailored for batteryless IoT nodes. We introduce a system architecture that dynamically adjusts the modulation order and data rate based on real-time channel conditions and the energy reserves of the batteryless device. Mathematical models for VLC channel estimation, energy consumption, and signal-to-noise ratio (SNR) based data rate adaptation are presented. The proposed adaptive approach aims to maximize communication efficiency and reliability, ensuring sustained operation even under fluctuating environmental and energy constraints.
\\end{abstract}

\\section{Introduction}
The proliferation of the Internet of Things (IoT) demands pervasive, long-lasting, and often maintenance-free sensor nodes. Batteryless IoT devices, powered solely by ambient energy harvesting, represent a critical advancement towards this goal \cite{sanislav2021energy}. Among various communication technologies, Visible Light Communication (VLC) has emerged as a promising candidate for its energy efficiency, inherent security, and dual-use capability when integrated with solar panels \cite{gutierrez2026solar}. A single solar panel can simultaneously harvest energy from incident light and act as a photodetector for modulated light signals, significantly reducing the hardware footprint and power consumption compared to traditional radio frequency (RF) systems \cite{xu2017passivevlc}.

Recent work has demonstrated the feasibility of solar panel-based VLC for batteryless systems, showcasing reliable data reception with remarkably low power consumption in sleep states \cite{gutierrez2026solar}. However, a key limitation of current batteryless VLC implementations is their reliance on fixed data rates and modulation schemes. The performance of VLC links is highly dependent on environmental factors such as ambient light intensity, distance to the light source, and potential obstructions. These fluctuating conditions directly impact the received signal-to-noise ratio (SNR) and, consequently, the achievable data rate and reliability. A fixed-rate system will either be over-provisioned (wasting energy when conditions are good) or under-provisioned (failing to communicate when conditions degrade).

This paper addresses this challenge by proposing an adaptive VLC system for batteryless IoT nodes. Our objective is to dynamically adjust the VLC data rate and modulation scheme to match the prevailing channel conditions and the available energy budget, thereby optimizing throughput, energy efficiency, and communication reliability.

\\section{Background: Solar Panel-based VLC and Energy Harvesting}
The fundamental principle of solar panel-based VLC involves using the photovoltaic effect for both energy conversion and signal detection. A solar panel generates a current proportional to the incident light intensity. By modulating the intensity of an LED light source at high frequencies, these variations can be detected as AC signals superimposed on the DC photocurrent generated by ambient light.

The architecture typically involves a solar panel, an energy harvesting management unit (EHMU), a capacitor for energy storage, and a low-power microcontroller unit (MCU). The EHMU charges the capacitor, providing a stable voltage to the MCU. For VLC reception, a lightweight analog front-end filters out the DC component and conditions the AC signal for the MCU's analog-to-digital converter (ADC). The reception is often synchronized with the open-circuit phase of the harvester to minimize interference, as highlighted in \cite{gutierrez2026solar}.

The maximum achievable data rate ($R$) over a VLC channel is fundamentally limited by the Shannon-Hartley theorem, which states:
$$ R = B \\log_2(1 + S/N) $$
where $B$ is the bandwidth of the channel and $S/N$ is the signal-to-noise ratio (SNR). In VLC, the SNR can vary significantly due to:
\\begin{itemize}
    \\item \textbf{Ambient Light Intensity:} High ambient light (e.g., sunlight) increases the DC photocurrent, potentially increasing noise and reducing the relative strength of the modulated signal.
    \\item \textbf{Distance and Angle:} The received optical power decreases with distance and depends on the angle of incidence from the LED source.
    \\item \textbf{Obstructions:} Partial or full blockages reduce received optical power.
    \\item \textbf{LED Characteristics:} The intensity and modulation capabilities of the transmitting LED.
\\end{itemize}
Fixed modulation schemes (e.g., On-Off Keying (OOK) at a single rate) fail to account for these dynamics, leading to either missed opportunities for higher throughput or unreliable communication when conditions worsen.

\\section{Adaptive VLC System Design}
We propose an adaptive VLC system where the batteryless IoT node continuously monitors its environment and internal state to adjust its communication parameters. The key components and their interactions are illustrated conceptually in Figure 1 (conceptual diagram, not rendered here for brevity).

\\subsection{System Architecture}
The adaptive system extends the batteryless VLC node architecture by incorporating:
\\begin{itemize}
    \\item \textbf{Channel Estimator:} Responsible for real-time estimation of the VLC channel quality, primarily through SNR.
    \\item \textbf{Energy Manager:} Monitors the capacitor's charge level and predicts available energy for communication tasks.
    \\item \textbf{Adaptive Modulator Controller:} Based on input from the Channel Estimator and Energy Manager, this module selects the optimal modulation scheme and data rate.
\\end{itemize}

The communication process would follow a feedback loop:
\begin{enumerate}
    \item The batteryless node (receiver) periodically estimates the channel conditions (SNR) using received preamble signals or dedicated pilot tones.
    \item It also monitors its internal energy reserves (capacitor voltage).
    \item This information is communicated back to the VLC transmitter (e.g., via a low-rate, robust uplink channel like BLE in a hybrid system, or through specific VLC feedback signals if duplex VLC is available).
    \item The transmitter, or the receiver itself if making autonomous decisions, then selects an appropriate modulation scheme (e.g., OOK, M-PPM, M-QAM) and data rate.
    \item The transmitter sends data using the chosen scheme.
\end{enumerate}

\\subsection{Mathematical Formulation for Adaptation}

\\subsubsection{SNR Estimation}
The received electrical signal power ($P_{sig}$) from the VLC source and the noise power ($P_{noise}$) are crucial for SNR estimation. The SNR at the receiver can be expressed as:
$$ SNR = \\frac{(\gamma P_{rec})^2}{P_{noise}} $$
where $\gamma$ is the responsivity of the solar panel acting as a photodetector, and $P_{rec}$ is the instantaneous received optical power. $P_{noise}$ typically includes shot noise from ambient light and thermal noise from the receiver circuitry.
The received optical power $P_{rec}$ from an LED source at distance $d$ can be modeled as:
$$ P_{rec} = I_L A_R \\cos(\\phi) $$
where $I_L$ is the irradiance from the LED, $A_R$ is the effective receiving area of the solar panel, and $\phi$ is the angle of incidence. The irradiance $I_L$ itself depends on the LED's luminous intensity and the distance.

\\subsubsection{Modulation Schemes and Data Rate}
For intensity modulation/direct detection (IM/DD) VLC systems, common modulation schemes include On-Off Keying (OOK), Pulse Position Modulation (PPM), and Quadrature Amplitude Modulation (QAM) with orthogonal frequency-division multiplexing (OFDM-VLC). Each scheme offers a different trade-off between spectral efficiency, energy efficiency, and complexity.

For a target Bit Error Rate (BER), the required SNR for different M-ary modulation schemes varies. For example, for M-PPM:
$$ BER \\approx \\frac{M}{2} Q\\left(\\sqrt{\\frac{R_b}{B_{eff}} \\frac{M}{2} \\frac{SNR}{ (e^{SNR}-1) / (SNR) }}\\right) $$
where $Q(x)$ is the Q-function, $R_b$ is the bit rate, and $B_{eff}$ is the effective bandwidth. More simply, for higher order QAM, the BER is approximated as:
$$ BER \\approx 2 \\frac{\\sqrt{M}-1}{\\sqrt{M}} Q\\left(\\sqrt{\\frac{3}{2(M-1)} SNR}\\right) $$
To achieve a target BER, a higher-order modulation (larger $M$) requires a higher SNR.

The adaptive algorithm selects a modulation order $M$ (and corresponding data rate $R_M$) based on the estimated SNR and the available energy. A simplified decision rule could be:
\\begin{itemize}
    \\item If $SNR < SNR_{threshold1}$: Use a very robust, low-rate scheme (e.g., OOK with significant error correction, or even go to sleep).
    \\item If $SNR_{threshold1} \\le SNR < SNR_{threshold2}$: Use OOK with a basic data rate.
    \\item If $SNR_{threshold2} \\le SNR < SNR_{threshold3}$: Use 2-PPM or a higher data rate OOK.
    \\item If $SNR_{threshold3} \\le SNR$: Use 4-PPM or low-order QAM (e.g., 4-QAM) for higher data rates.
\\end{itemize}
These thresholds ($SNR_{threshold1}, SNR_{threshold2}, SNR_{threshold3}$) would be predetermined based on desired BER targets for each modulation scheme.

\\subsubsection{Energy-Aware Adaptation}
In a batteryless system, the available energy is a critical constraint. The energy consumed per bit ($E_b$) for different modulation schemes and data rates will vary. The total energy required for a communication event of duration $T$ at power $P_{tx}$ is $E_{total} = P_{tx} T$. However, the receiver's energy consumption is dominated by the processing required for decoding. Higher-order modulation schemes generally require more complex decoding algorithms, leading to higher energy consumption per bit for the MCU.

The decision for data rate and modulation should also factor in the current capacitor voltage ($V_{cap}$). Let $E_{current} = \\frac{1}{2} C V_{cap}^2$ be the stored energy. The system should only attempt a communication mode if $E_{current}$ is above a certain threshold $E_{min\_op}$ and, ideally, sufficient to complete the transmission/reception while leaving enough energy for essential functions.

An adaptive policy could be formulated as an optimization problem:
$$ \\text{Maximize } R(SNR, M) \\text{ subject to } BER \\le BER_{target} \\text{ and } E_{comm}(M, R) \\le E_{available} $$
where $E_{comm}$ is the energy required for communication using a given modulation $M$ and rate $R$.

\\section{Performance Metrics and Evaluation}
To evaluate the proposed adaptive VLC system, the following metrics would be crucial:
\\begin{itemize}
    \\item \textbf{Average Throughput:} The amount of data successfully transmitted over a given period, especially under varying channel conditions.
    \\item \textbf{Energy Efficiency:} Bits per Joule ($bits/Joule$), which quantifies how much data is transmitted for a given amount of energy consumed by the communication module.
    \\item \textbf{Reliability (BER):} The measured bit error rate, ensuring it stays below a target threshold across different conditions.
    \\item \textbf{System Uptime/Longevity:} How long the batteryless device can sustain communication tasks before running out of energy or requiring a recharge cycle.
\\end{itemize}
Evaluation could be performed through detailed simulations accounting for varying ambient light profiles, distance changes, and energy harvesting rates, followed by prototype implementation and real-world testing.

\\section{Future Work}
Building upon the conceptual framework presented, several avenues for future research emerge:
\\begin{itemize}
    \\item \textbf{Development of Robust Channel Estimation Techniques:} For batteryless devices, complex channel estimation can be energy-intensive. Investigating lightweight, yet accurate, SNR estimation algorithms specifically designed for low-power constraints is crucial.
    \\item \textbf{Reinforcement Learning for Adaptation Policies:} Explore the use of reinforcement learning agents that can learn optimal adaptive policies over time, considering the complex interplay between SNR, energy levels, application requirements, and historical performance. This could lead to more nuanced and predictive adaptation strategies.
    \\item \textbf{Hybrid VLC/RF Adaptive Systems:} Further investigate how the adaptive VLC can seamlessly integrate with a low-power RF fallback (like BLE, as in the original work) to create an even more robust and ubiquitous communication solution, switching between modes based on an extended set of environmental and energy parameters.
    \\item \textbf{Impact of Receiver Mobility:} Analyze and design adaptive schemes for scenarios where the batteryless node might experience limited mobility, leading to more dynamic changes in line-of-sight and incident angles.
\\end{itemize}

\\section{Conclusion}
This paper outlines a framework for adaptive VLC data rate and modulation schemes for batteryless IoT systems, directly addressing the limitations of fixed-rate approaches in dynamic environments. By integrating real-time channel condition monitoring and energy awareness into the communication protocol, the proposed system aims to significantly enhance throughput, energy efficiency, and reliability. The mathematical foundations for SNR estimation and energy-constrained modulation selection are presented, laying the groundwork for future research and practical implementations of truly resilient and self-sustaining IoT communication nodes.

\\begin{thebibliography}{9}

\\bibitem{gutierrez2026solar}
Gutierrez, J. F., Nguyen, N., Quintero, J. M., \& Gomez, A. (2026). \\textit{Solar Panel-based Visible Light Communication for Batteryless Systems}. arXiv preprint arXiv:2601.04190. \\href{https://arxiv.org/pdf/2601.04190v1}{PDF}

\\bibitem{sanislav2021energy}
Sanislav, T., Mois, G. D., Zeadally, S., \& Folea, S. C. (2021). Energy harvesting techniques for Internet of Things (IoT). \\textit{IEEE Access}, \\textit{9}, 39530-39549. \\href{https://doi.org/10.1109/ACCESS.2021.3064066}{PDF}

\\bibitem{xu2017passivevlc}
Xu, X., Shen, Y., Yang, J., Xu, C., Shen, G., Chen, G., \& Ni, Y. (2017). PassiveVLC: Enabling practical visible light backscatter communication for battery-free IoT applications. In \\textit{Proceedings of the 23rd Annual International Conference on Mobile Computing and Networking (MobiCom �17)} (pp. 180-192). \\href{https://doi.org/10.1145/3117811.3117843}{PDF}

\\end{thebibliography}

\\end{document}
